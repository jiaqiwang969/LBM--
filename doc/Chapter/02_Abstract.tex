%==============================================================================
\documentclass[MathematicsNumericsDerivationsAndOpenFOAM.tex]{subfiles}
\begin{document}
%==============================================================================
%
%
\textit{Copyright Tobias Holzmann, \today}
%
%
\vspace{1cm}



\subsubsection*{This book covers the following subjects}
%
%
%
%
    This book primarily collects aspects of mathematics, numerics, and
    derivations used in the field of computational fluid dynamics (CFD) and
    covers a few aspects regarding the open-source project named \OF.
    The author of the book, Tobias Holzmann, tried to write understandable and
    transparently.
%
%
%
%
%
\subsubsection*{Differences in the release versions}
%
%
	The release notes of the book are available at \url{https://holzmann-cfd.com}.
%
%
%
%
\subsubsection*{Acknowledgment}
%
%
%
%
    A former colleague of Tobias, Dr. Alexander Vakhrushev, is acknowledged for
    the exciting discussions during his time at the Montanuniversität Leoben
    as well as the deep insight into different topics such as mathematics and
    programming in c++. Furthermore, all people listed on Tobias website, which
    gave knowledgeable remarks to the book, are warmly acknowledged, and Tobias
    is thankful for the time these people investigated. Additional, Tobias
    acknowledge his lovely former wife, Andrea Elisabeth Jall, for all support
    she gave to him and also the improvements and extensions she made to the book.
%
%
%

\subsubsection*{How to cite?}
%
%
    The citation format depends on the journal one is using or to your personal
    style. The following citation is an example based on the format that
    is used in this document:
    \vspace{0.5cm}
%
%
\newline
%
%
%
    Tobias Holzmann. \textit{Mathematics, Numerics, Derivations and OpenFOAM(R)},
    Holzmann CFD, Release 7-DEV, URL \url{https://Holzmann-cfd.com},
    DOI: 10.13140/RG.2.2.27193.36960.

%
%
%
\newpage
%
%
\section*{Outline}
%
%
%
    This book gives an introduction to the underlying mathematics used in the
    field of computational fluid dynamics. After presenting the fundamental mathematic
    aspects, all conservation equations are derived using a finite volume
    element, d$V$. In the beginning, the derivation of the mass and momentum
    equation is described. Subsequently, all kinds of the energy equation are
    discussed and presented, namely the kinetic energy, internal energy, total
    energy, and the enthalpy equation.


    After all relevant equations are derived, the general governing equation is
    introduced, and it is demonstrated how one can derive different equations
    while using the general one.


    The subsequent chapters discuss the definition of the shear-rate tensor
    $\boldsymbol \tau$ for Newtonian fluids which is followed by a discussion
    regarding the analogy of the Cauchy stress tensor $\boldsymbol \sigma$,
    the shear-rate tensor $\boldsymbol \tau$ as well as the pressure $p$.
    At the end of the first part, all equations are given on a
    \textit{one page summary} page.
%
%
\\~\\
%
%
    Because engineering applications are mostly turbulent, the Reynolds-Averaging
    methods are presented and explained.
    Subsequently, the incompressible \hyphenation{Reynolds-Averaged-Navier-Stokes}
    equations are derived, and finally, the closure problem is discussed in
    detail. Here, the Reynolds-Stress equation --- which is fully derived in
    the appendix --- and the analogy to the Cauchy stress tensor is presented.


    To close the subject of turbulent flows, the eddy-viscosity theory is
    introduced, and the equation for the turbulent kinetic energy $k$ and
    turbulent dissipation $\epsilon$ are deducted. The topic ends with a
    brief description of the derivation for the compressible
    Navier-Stokes-Equations equations and its difficulties and validity.
%
%
\\~\\
%
%
    The last chapters of the book are related to the open-source toolbox \OF.
    First, a detailed explanation of the implementation of the shear-rate tensor
    calculation in \OF is presented. Here, the c++ code is investigated and
    compared against the mathematical expression.


    Subsequently, a more general discussion of the different pressure-momentum
    coupling algorithms are presented. Here, primarily the \PIMPLE-algorithm is
    explained by analyzing an \OF case.


    It follows an entire chapter regarding the matrix handling in \OF is
    given. The reader will get information regarding the constructions of
    the matrix systems and its units in \OF while analyzing the c++ code.
%
%
\\~\\
%
%
    At the end an overview for \OF beginners who are seeking for tutorials
    and some other useful information and websites is presented.
%
%
%==============================================================================
\end{document}
%==============================================================================
