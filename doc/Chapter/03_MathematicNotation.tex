%==============================================================================
\documentclass[MathematicsNumericsDerivationsAndOpenFOAM.tex]{subfiles}
\begin{document}
%==============================================================================


\chapter{Basic Mathematics}
\label{CHAPTER::Mathematics}
\pagenumbering{arabic}
%
%
%

	在计算流体力学领域,最基本的一点是要理解方程和数学。如果一个人要在一个确定的软件或工具箱中实现、重新排序或操作方程,就需要这种知识。有许多表示方程的方法,因此在本章中简要收集了一些基本的数学知识。 \cite{JasakPhD, Rappaz, ProgrammersGuide}、 \cite{Moukalled15}和其他知名文献中也描述了数学的魅力。


在数值模拟领域,我们要处理的是等级为$n$的$\textbf{T}^n$。一个  \texttt{tensor} 代表任何场。一个场可以代表一个标量、一个矢量,或者众所周知的代表一个矩阵的张量------在计算流体力学领域,在有限体积法,通常是一个3乘3的矩阵;这里,张量的秩是2。为了保持清晰,我们使用下面的定义,它等同于\cite{ProgrammersGuide}。


%
%
	\vspace{0.5cm}

	0阶 \texttt{tensor} \textbf{T}$^0$ := 标量 $a$

	1阶 \texttt{tensor} \textbf{T}$^1$ := 矢量 $\textbf{a}$

	2阶 \texttt{tensor} \textbf{T}$^2$ := 张量 $\textbf{T}$ (通常为 3x3 矩阵)

	3阶 \texttt{tensor} \textbf{T}$^3$ := 张量 $T_{ijk}$

	~\\
%
%
	如果一个张量的秩大于零,该张量就会被写成黑体字。用粗体符号/字母书写。等级大于2的张量 在本书介绍的大多数方程中都不需要。唯一的例外是 是雷诺-应力方程的推导。
%
%
\section{推导的基本规则}
%
%
	流体动力学中的主控方程是偏微分方程。因此,现在介绍一下操作和分析这些方程所需的规则的摘要。


  	考虑到两个数量$\phi$和$\chi$的总和,这些数量是根据$\tau$得出的。根据$\tau$,导数可以被拆分成以下几个部分。
%
%
\begin{equation}
 \frac{\partial (\phi + \chi)}{\partial \tau}
 =
 \frac{\partial \phi }{\partial \tau} + \frac{\partial \chi}{\partial \tau} ~.
\end{equation}
%
%

如果我们有两个量的乘积的导数,就有可能使用 \texttt{product rule}来拆分这个项。换句话说,我们必须保持一个量不变,同时推导出另一个量。

%
%
\begin{equation}
 \frac{\partial \phi \chi}{\partial \tau}
 =
 \chi \frac{\partial \phi }{\partial \tau}
+
 \phi \frac{\partial \chi}{\partial \tau}
 \label{EQUATION::productRuleCartesian} ~.
\end{equation}
%
%

如果两个以上的量相乘,乘积规则仍然成立。比如说

%
%
\begin{equation}
 \frac{\partial \phi \chi \zeta}{\partial \tau}
 =
 \chi \frac{\partial \phi\zeta }{\partial \tau}
+
 \phi \frac{\partial \chi\zeta}{\partial \tau}
+
 \zeta \frac{\partial \phi \chi}{\partial \tau}
\end{equation}
%
%

	一个常量$mathop{C}$可以在一个导数的内部或外部取用,而不受任何约束。导数的内部或外部,不受任何限制。
%
%
\begin{equation}
 \frac{\partial \mathop{C} \phi \chi}{\partial \tau}
=
 \mathop{C} \frac{\partial \phi \chi}{\partial \tau} ~.
\end{equation}
%
%
%
%
%
\section{Einsteins 求和约定}
%
%
	对于矢量和张量方程来说,有几种符号的选择。最长但最清晰的符号是笛卡尔的符号。如果方程包含几个可以求和的类似项,这种符号可以通过应用Einsteins求和约定来简化。假设以下任意变量$phi_i$在$x$、$y$和$z$方向的导数之和(类似于质量守恒方程),笛卡尔形式由以下公式给出
%
%
\begin{equation*}
 \frac{\partial \phi_x}{\partial x}
+\frac{\partial \phi_y}{\partial y}
+\frac{\partial \phi_z}{\partial z} ~.
\end{equation*}
%
%
	为了简化这个方程,可以应用Einsteins求和约定。通常情况下,为了保持清晰,求和符号$\sum$被忽略了
%
%
\begin{equation}
  \sum_{i} \frac{\partial \phi_i}{\partial x_i}
=
  \frac{\partial \phi_i}{\partial x_i} ~~~~~~ i=x,y,z ~.
\end{equation}
%
%
    一个更复杂的例子证明了Einsteins 求和约定的优势,那就是动量方程的对流项(现在没有必要知道这个项的含义)。由于动量是一个矢量的事实,每个方向的三个单项给定为
%
%
\begin{equation*}
 \frac{\partial u_x u_x}{\partial x}
 +
 \frac{\partial u_y u_x}{\partial y}
 +
    \frac{\partial u_z u_x}{\partial z}~~~~~~ \mathrm{x-direction},
\end{equation*}
%
%
\begin{equation*}
 \frac{\partial u_x u_y}{\partial x}
 +
 \frac{\partial u_y u_y}{\partial y}
 +
    \frac{\partial u_z u_y}{\partial z}~~~~~~ \mathrm{y-direction},
\end{equation*}
%
%
\begin{equation*}
 \frac{\partial u_x u_z}{\partial x}
 +
 \frac{\partial u_y u_z}{\partial y}
 +
    \frac{\partial u_z u_z}{\partial z}~~~~~~ \mathrm{z-direction}.
\end{equation*}
%
%
按照Einsteins 求和约定,这九个术语可以表示为如下
%
%
\begin{equation}
 \sum_j \sum_i \frac{\partial u_i u_j}{\partial x_i} e_j =  \frac{\partial u_i u_j}{\partial x_i} e_j ~~~~~~i=x,y,z; ~j=x,y,z~.
\end{equation}
%
%
    符号$e_j$在下一节定义,代表$x$、$y$和$z$方向的单位向量。如果我们忽略单位向量,我们最终会得到所有九项的单一求和。然而,正如人们所看到的,我们有三个与坐标$x$、$y$和$z$有关的单一求和。因此,单位矢量必须被加上。爱因斯坦的求和惯例在文献中被广泛使用。因此,了解它的含义和应用方式是非常重要的。
%
%
    请记住,在大多数文献中,变量$i$和$j$被设定为$1$、$2$和$3$,分别代表三个坐标方向,比较下一节。
%
%
\section{张量数学约定}
%
%
	处理方程的一个常见和简单的方法是使用矢量符号而不是Einsteins求和约定。矢量符号需要特殊的数学知识。因此,现在简要介绍一下适用于标量、向量和张量的不同操作。

	为此,现在定义了任意量:一个标量$\phi$,两个向量$\textbf{a}$和$\textbf{b}$以及一个张量$\textbf{T}$。
%
%
\begin{equation*}
  \textbf{a}
=
  \left(
  \begin{matrix}
    a_x \\ a_y \\ a_z
  \end{matrix}
  \right)
=
  \left(
  \begin{matrix}
    a_1 \\ a_2 \\ a_3
  \end{matrix}
  \right)
  ~,~~~~~
  \textbf{b}
=
  \left(
  \begin{matrix}
    b_x \\ b_y \\ b_z
  \end{matrix}
  \right)
=
  \left(
  \begin{matrix}
    b_1 \\ b_2 \\ b_3
  \end{matrix}
  \right)
  ~,
\end{equation*}
%
%
\begin{equation*}
  \textbf{T}
=
  \left[
  \begin{matrix}
   T_{xx} ~ ~ ~ ~ T_{xy} ~ ~ ~ ~ T_{xz} \\
   T_{yx} ~ ~ ~ ~ T_{yy} ~ ~ ~ ~ T_{yz} \\
   T_{zx} ~ ~ ~ ~ T_{zy} ~ ~ ~ ~ T_{zz}
  \end{matrix}
  \right]
=
  \left[
  \begin{matrix}
   T_{11} ~ ~ ~ ~ T_{12} ~ ~ ~ ~ T_{13} \\
   T_{21} ~ ~ ~ ~ T_{22} ~ ~ ~ ~ T_{23} \\
   T_{31} ~ ~ ~ ~ T_{32} ~ ~ ~ ~ T_{33}
  \end{matrix}
  \right] ~.
\end{equation*}
%
%

	我们使用数字指数($1$,$2$,$3$)或空间分量($x$,$y$,$z$)定义单位向量$\textbf{e}_i$和单位矩阵$\textbf{I}$,如下
%
%
\begin{equation*}
\textbf{e}_1 = \textbf{e}_x = \left( \begin{matrix}   1 \\ 0 \\ 0  \end{matrix}  \right)
~,~~~
\textbf{e}_2= \textbf{e}_y = \left( \begin{matrix}   0 \\ 1 \\ 0  \end{matrix}  \right)
~,~~~
\textbf{e}_3= \textbf{e}_z = \left( \begin{matrix}   0 \\ 0 \\ 1  \end{matrix}  \right)
~,~~~
\textbf{I} = \left[ \begin{matrix}   1 & 0 & 0 \\ 0 & 1 & 0 \\ 0 & 0 & 1  \end{matrix}  \right] ~.
\end{equation*}
%
%
%
\subsubsection{基本运算}
%
%
\begin{itemize}
    \item 标量$\phi$与矢量$\textbf{b}$相乘的结果是一个矢量,并且是可交换的。这也适用于标量$\phi$与张量$\textbf{T}$的乘法。
\end{itemize}
%
%
\begin{equation}
  \phi \textbf{b} =
  \left(
  \begin{matrix}
    \phi b_x \\ \phi b_y \\ \phi b_z
  \end{matrix}
  \right)
  ,~~~~~
  \phi\textbf{T}
=
  \left[
  \begin{matrix}
   \phi T_{xx} ~ ~ ~ ~ \phi T_{xy} ~ ~ ~ ~ \phi T_{xz} \\
   \phi T_{yx} ~ ~ ~ ~ \phi T_{yy} ~ ~ ~ ~ \phi T_{yz} \\
   \phi T_{zx} ~ ~ ~ ~ \phi T_{zy} ~ ~ ~ ~ \phi T_{zz}
  \end{matrix}
  \right].
\label{EQUATION::simple}
\end{equation}
%
%
%
%
\subsubsection{内积}
%
%
\begin{itemize}
    \item 两个向量$\textbf{a}$和$\textbf{b}$的内积产生一个标量$\phi$,并且是可交换的。这个操作用点号$\bullet$表示。
\end{itemize}
%
%
\begin{equation}
 \phi = \textbf{a} \bullet \textbf{b} = \vA \tr \vB = \sum_{i=1}^{3} a_i b_i ~.
 \label{EQUATION::innerProductVV}
\end{equation}
%
%
%
\begin{itemize}
    \item  向量$\textbf{a}$与张量$\textbf{T}$的内积产生一个向量$\textbf{b}$,如果张量是非对称的,则是非交换的。
\end{itemize}
%
%
\begin{equation}
 \textbf{b} = \textbf{T} \bullet \textbf{a} =
 \sum_{i=1}^{3}\sum_{j=1}^{3} T_{ij} a_j   \textbf{e}_i
 =
  \left(
  \begin{matrix}
   T_{11} a_1 ~ ~ + ~ ~ T_{12} a_2 ~ ~ + ~ ~ T_{13} a_3 \\
   T_{21} a_1 ~ ~ + ~ ~ T_{22} a_2 ~ ~ + ~ ~ T_{23} a_3 \\
   T_{31} a_1 ~ ~ + ~ ~ T_{32} a_2 ~ ~ + ~ ~ T_{33} a_3
  \end{matrix}
  \right).
  \label{EQUATION::innerProductTV}
\end{equation}
%
%
\begin{equation}
 \textbf{b} = \textbf{a} \bullet \textbf{T}
 =
 \tT \tr \bullet \vA
 =
 \sum_{i=1}^{3}\sum_{j=1}^{3} a_j T_{ji}  \textbf{e}_i
 =
  \left(
  \begin{matrix}
   a_1 T_{11} ~ ~ + ~ ~ a_2 T_{21} ~ ~ + ~ ~ a_3 T_{31} \\
   a_1 T_{12} ~ ~ + ~ ~ a_2 T_{22} ~ ~ + ~ ~ a_3 T_{32} \\
   a_1 T_{13} ~ ~ + ~ ~ a_2 T_{23} ~ ~ + ~ ~ a_3 T_{33}
  \end{matrix}
  \right),
  \label{EQUATION::innerProductVT}
\end{equation}
%
%
\begin{itemize}
   \item[] 如果给定一个对称矩阵, 即 $\textbf{T}_{ij} = \textbf{T}_{ji}$
   则 $\vA\bullet\tT = \tT \bullet \vA$.
\end{itemize}
%
%
%\begin{itemize}
% \item The inner product of two tensors $\textbf{T}$ and $\textbf{S}$ results in a tensor $\textbf{V}$ and is non-commutative:
%\end{itemize}
%
%
%\begin{equation}
% \textbf{V} = \textbf{T} \bullet \textbf{S} ~ ~ \Longleftrightarrow ~ ~V_{ij} =  T_{ik} S_{kj}
% \label{EQUATION::innerProductTT}
%\end{equation}
%
%
%
%
\subsubsection{双内积}
%
%
\begin{itemize}
   \item 两个张量$\textbf{T}$和$\textbf{S}$的双内积的结果是一个标量$\phi$,并且是可交换的。它将被表示为冒号 $\boldsymbol \colon$ 符号:
\end{itemize}
%
%
\begin{multline}
 \phi = \textbf{T} \boldsymbol \colon \textbf{S} = \sum_{i=1}^{3}\sum_{j=1}^{3} T_{ij} S_{ij}
=
 T_{11} S_{11} + T_{12} S_{12} + T_{13} S_{13} + T_{21} S_{21} \\
+
 T_{22} S_{22} +
 T_{23} S_{23} + T_{31} S_{31} + T_{32} S_{32} + T_{33} S_{33} ~.
\label{EQUATION::doubleInnerProduct}
\end{multline}
%
%
%
%
\subsubsection{外积}
%
%
\begin{itemize}
    \item 两个向量$\textbf{a}$和$\textbf{b}$的外积,也被称为并矢(dyadic),结果是一个张量,是非可交换的,用并矢符号$\otimes$表示。
\end{itemize}
%
%
%
\begin{equation}
  \textbf{T} = \textbf{a} \otimes \textbf{b} = \textbf{a} \textbf{b}^T =   \left[
  \begin{matrix}
   a_xb_x ~ ~ ~ ~ ~ a_xb_y ~ ~ ~ ~ ~ a_xb_z \\
   a_yb_x ~ ~ ~ ~ ~ a_yb_y ~ ~ ~ ~ ~ a_yb_z \\
   a_zb_x ~ ~ ~ ~ ~ a_zb_y ~ ~ ~ ~ ~ a_zb_z
  \end{matrix}
  \right].
   \label{EQUATION::dyadic}
\end{equation}
%
%
	在大多数文献中,我们会发现,为了简洁起见,忽略了并矢标记 $\otimes$,如下所示
%
%
\begin{equation}
  \textbf{a}\textbf{b}~.
  \label{EQUATION::outerProduct}
\end{equation}
%
%
	请记住,这两种变体都在文献中使用,而最后一种更常见,但第一种在数学上更常见。在本书中,我们使用方程的定义(\ref{EQUATION::dyadic}),以便与数学保持一致。
%
%
%
%
\subsubsection{微分运算符}
%
%
	在矢量符号中,一个变量(标量、矢量或张量)的空间导数是用Nabla算子$\nabla$进行的。它包含直角坐标系中$x$、$y$和$z$的三个空间导数。
%
%
$$
  \nabla
=
  \left(
  \begin{matrix}
    \frac{\partial}{\partial x} \\
    \frac{\partial}{\partial y} \\
    \frac{\partial}{\partial z}
  \end{matrix}
  \right)
=
  \left(
  \begin{matrix}
    \frac{\partial}{\partial x_1} \\
    \frac{\partial}{\partial x_2} \\
    \frac{\partial}{\partial x_3}
  \end{matrix}
  \right) .
$$
%
%
%
%
\subsubsection{梯度运算符}
%
%
\begin{itemize}
    \item 标量$\phi$的梯度会产生一个向量$\textbf{a}$:
\end{itemize}
%
%
\begin{equation}
 \operatorname{grad}\phi = \nabla \phi
=
  \left(
  \begin{matrix}
    \frac{\partial \phi}{\partial x} \\
    \frac{\partial \phi}{\partial y} \\
    \frac{\partial \phi}{\partial z}
  \end{matrix}
  \right) .
  \label{EQUATION::gradientScalar}
\end{equation}
%
%
\begin{itemize}
    \item 一个矢量$\textbf{b}$的梯度会产生一个张量$\textbf{T}$:
\end{itemize}
%
%
\begin{equation}
 \operatorname{grad}\textbf{b} = \nabla \otimes \textbf{b}
=
  \left[
  \begin{matrix}
   \frac{\partial}{\partial x}b_x ~ ~ ~ ~ ~ \frac{\partial}{\partial x} b_y ~ ~ ~ ~ ~ \frac{\partial}{\partial x} b_z \\
   \frac{\partial}{\partial y}b_x ~ ~ ~ ~ ~ \frac{\partial}{\partial y} b_y ~ ~ ~ ~ ~ \frac{\partial}{\partial y} b_z \\
   \frac{\partial}{\partial z}b_x ~ ~ ~ ~ ~ \frac{\partial}{\partial z} b_y ~ ~ ~ ~ ~ \frac{\partial}{\partial z} b_z
  \end{matrix}
  \right] .
   \label{EQUATION::gradientVector}
\end{equation}
%
%
我们看到,这个操作是Nabla算子(一个特定的向量)和一个任意向量$\textbf{b}$的外积。因此,它通常被写成
%
%
\begin{equation}
  \nabla \textbf{b} ~.
\end{equation}
%
%
    在本书中,我们使用第一种符号(dyadic符号),以便在数学中保持一致。梯度运算(gradient operation)使张量的秩增加了1,因此,我们可以将其应用于任何张量领域。
%
%
%
%
\subsubsection{散度运算符}
%
%
\begin{itemize}
    \item 矢量$\textbf{b}$的散度结果是一个标量$\phi$,由Nabla算子和点号的组合表示,$\nabla \bullet$。
\end{itemize}
%
%
\begin{equation}
 \operatorname{div} \textbf{b} = \nabla \bullet \textbf{b} = \sum_{i=1}^{3} \frac{\partial}{\partial x_i} b_i
=
  \frac{\partial b_1}{\partial x_1} + \frac{\partial b_2}{\partial x_2} + \frac{\partial b_3}{\partial x_3} ~.
  \label{EQUATION::divVector}
\end{equation}
%
%
\begin{itemize}
    \item 张量 $\textbf{T}$的散度为向量 $\textbf{b}$:
\end{itemize}
%
%
\begin{equation}
  \operatorname{div} \textbf{T} = \nabla \bullet \textbf{T}
=
  \sum_{i=1}^{3}\sum_{j=1}^{3} \frac{\partial}{\partial x_j} T_{ji} \textbf{e}_i
=
  \left[
  \begin{matrix}
   \frac{\partial T_{11}}{\partial x_1} ~ ~ + ~ ~ \frac{\partial T_{21}}{\partial x_2} ~ ~ + ~ ~ \frac{\partial T_{31}}{\partial x_3} \\
   \frac{\partial T_{12}}{\partial x_1} ~ ~ + ~ ~ \frac{\partial T_{22}}{\partial x_2} ~ ~ + ~ ~ \frac{\partial T_{32}}{\partial x_3} \\
   \frac{\partial T_{13}}{\partial x_1} ~ ~ + ~ ~ \frac{\partial T_{23}}{\partial x_2} ~ ~ + ~ ~ \frac{\partial T_{33}}{\partial x_3}
  \end{matrix}
  \right] .
  \label{EQUATION::divTensor}
\end{equation}
%
%
    散度运算的结果$\textit{减少}$张量的阶数为1。因此,将这个算子应用在一个\texttt{标量}是没有意义的。
%
%
%
%
\subsubsection{散度运算的乘法}
%
%
	如果我们在一个散度项内有一个乘积,我们可以用乘积规则来拆分这个项。基于算子内部的$\texttt{张量}$等级,我们必须应用不同的规则,现在介绍一下
%
%
\begin{itemize}
    \item 矢量$\textbf{a}$和标量$\phi$的乘积的发散可以按如下方式拆分,结果是一个标量。
\end{itemize}
%
%
\begin{equation}
 \nabla \bullet (\textbf{a} \phi)  =  \underbrace{\textbf{a} \bullet \nabla \phi}_{\mathrm{Eqn.~} (\ref{EQUATION::innerProductVV})} + \underbrace{\phi \nabla \bullet \textbf{a}}_{\mathrm{simple~multiplication}}~.
 \label{EQUATION::productRuleVS}
\end{equation}
%
%
\begin{itemize}
    \item 两个向量 $\textbf{a}$ 和 $\textbf{b}$ 的外积(dyadic product)的拆分可以按如下方式,结果是一个向量。
\end{itemize}
%
%
\begin{equation}
 \nabla \bullet (\textbf{a} \otimes \textbf{b})  =  \underbrace{\textbf{a} \bullet \nabla \otimes \textbf{b}}_{\mathrm{Eqn.~}(\ref{EQUATION::innerProductVT})} + \underbrace{\textbf{b} \nabla \bullet \textbf{a}}_{\mathrm{Eqn.~}(\ref{EQUATION::simple})} ~.
 \label{EQUATION::productRuleVV}
\end{equation}
%
%
\begin{itemize}
    \item 张量$\textbf{T}$和矢量$\textbf{b}$的内积的散度可以按如下方式拆分,结果是一个标量。
\end{itemize}
%
%
\begin{equation}
 \nabla \bullet (\textbf{T} \bullet \textbf{b})  =  \underbrace{\textbf{T} \boldsymbol \colon \nabla \otimes \textbf{b}}_{\mathrm{Eqn.~} (\ref{EQUATION::doubleInnerProduct})} + \underbrace{\textbf{b} \bullet \nabla \bullet \textbf{T}}_{\mathrm{Eqn.~}(\ref{EQUATION::innerProductVV})} ~.
 \label{EQUATION::productRuleTV}
\end{equation}
%
%
%
%
\subsection{物质导数}
%
%
      一个任意量$\phi$的物质导数的定义,在流体动力学领域定义为:
%
%
\begin{equation}
    \frac{\mathrm{D}\phi}{\mathrm{D}t} = \frac{\partial \phi}{\partial t} + \underbrace{\textbf{U}\bullet \nabla \phi}_{\mathrm{inner~product}} ~,
    \label{EQUATION::totalDerivative}
\end{equation}
%
%
	其中$\textbf{U}$代表速度矢量。方程中的最后一项(\ref{EQUATION::totalDerivative})表示内积。根据$\phi$的数量(标量、矢量、张量等),必须应用右侧(RHS)第二项的正确数学表达。给出的例子。
%
%
\begin{itemize}
    \item 如果 $\phi$是标量, 需要应用方程 (\ref{EQUATION::innerProductVV}),
    \item 如果 $\phi$ 是矢量, 需要应用方程 (\ref{EQUATION::innerProductVT}).
\end{itemize}
%
%
%
%
\subsubsection{物质导数概述}
%
%

	物质导数是用来表示非守恒方程的。换句话说,每个守恒方程都可以用连续性方程改变为非守恒方程。在文献中,人们开始使用物质导数推导方程,并使用连续性方程将非守恒的方程扩展为守恒的方程。个人认为,更好的方法是首先推导出守恒方程,然后使用连续性方程,得到非保守形式。$\textit{Why}$? 对我来说,这更容易理解。


	这两个方程的区别在于参考框架。在守恒表示中,我们有欧拉表达式,对于非守恒方程,它是拉格朗日表达式。


	如果你有以非守恒形式的方程开始的文献,下面的章节应该有助于理解下面的扩展(目前没有必要理解这个方程)。
%
%
\begin{itemize}
    \item 不可压:
\end{itemize}
%
%
\begin{equation}
    \frac{\mathrm{D}\phi}{\mathrm{D}t}
=
    \underbrace
    {
        \frac{\partial \phi}{\partial t}
      + \textbf{U}\bullet \nabla \phi
    }_{\mathrm{non-conserved}}
 ~~~
     \boxed
     { + ~
        \phi
        \underbrace
        {
            \vphantom{\frac{1}{2}}\left( \nabla \bullet (\textbf{U})\right)
        }_{\mathrm{continuity~=~0}}
     } ~.
\end{equation}
%
%
\begin{itemize}
    \item 可压:
\end{itemize}
%
%
\begin{equation}
   \rho \frac{\mathrm{D}\phi}{\mathrm{D}t}
=
    \underbrace
    {
        \rho\left[\frac{\partial \phi}{\partial t}
      + \textbf{U}\bullet \nabla \phi\right]
    }_{\mathrm{non-conserved}}
 ~~~
    \boxed
    { + ~
        \phi
        \underbrace
        {
            \left( \frac{\partial \rho}{\partial t}
          + \nabla \bullet (\rho \textbf{U})\right)
        }_{\mathrm{continuity~=~0}}
    } ~.
\end{equation}
%
%
	连续性方程(右手边第二项)乘以数量$\phi$的原因来自乘积法则,它被应用于对流项。在推导出动量方程并将保守形式转化为非保守形式后,这一声明将变得清晰。
%
%
%
%
\subsection{矩阵运算、偏应力张量、球应力张量}
%
%
	在数值模拟领域,我们要处理的是一些由矩阵表示的量,如应力张量。矩阵所代表的数量,如应力张量。因此,一些基本的 现在介绍一些基本的数学表达式和操作方法。

	每个应力张量$\textbf{A}$都可以分成偏应力张量$\textbf{A}^\text{dev}$和球应力张量$\textbf{A}^\text{hyd}$部分。
%
%
\begin{equation}
  \textbf{A}
=
  \textbf{A}^\text{hyd}
+
  \textbf{A}^\text{dev}~.
  \label{EQUATION::deviatoricHydrostatic}
\end{equation}
%
%
	球应力张量部分可以表示为标量或矩阵,并通过使用迹算子来定义。如果想计算标量,我们使用以下定义。
%
%
\begin{equation}
  A^\rmm{hyd}
=
  \frac{1}{3}\operatorname{tr}(\textbf{A})
=
  \frac{1}{3} \sum_{i=1}^{n}(a_{ii}) ~.
  \label{EQUATION::hydrostaticScalar}
\end{equation}
%
%
	运算符$\texttt{tr}$表示迹运算符,适用于矩阵。这个算子简单地将对角线元素加起来。然而,矩阵$\textbf{A}$的球应力部分的正确数学表达式是:
%
%
%
\begin{equation}
  \textbf{A}^\rmm{hyd}
=
  A^\rmm{hyd} \textbf{I}
=
  \frac{1}{3}\operatorname{tr}(\textbf{A})\textbf{I}
=
  \frac{1}{3} \sum_{i=1}^{n}(a_{ii})\textbf{I} ~.
  \label{EQUATION::hydrostaticMatrix}
\end{equation}
%
%
%
	偏应力张量 $\textbf{A}^\mathrm{dev}$ 表示为:
%
%
\begin{equation}
 \textbf{A}^\mathrm{dev}
=
 \textbf{A} - \textbf{A}^\mathrm{hyd}
=
 \textbf{A} - \frac{1}{3}\operatorname{tr}(\textbf{A})\textbf{I}~.
 \label{EQUATION::deviatoric}
\end{equation}
%
%
	\textbf{备注}: 偏应力部分是\textit{traceless}。因此,$\operatorname{tr}(\textbf{A}^\mathrm{dev}) = 0$; 请记住。迹算子是零标量而不是对角线元素。
	
%
%
%
%
%
\subsection{Gauss 定理}
%
%
	为了将方程从微分形式转化为积分形式(或反之),有必要知道高斯定理。该定理使我们能够建立任意体积元素的通量与体积元素上的发散算子之间的关系。
%
%
\begin{equation}
 \oint \textbf{a} \cdot \textbf{n} \mathrm{d}S
=
 \int (\nabla \bullet \textbf{a}) \mathrm{d}V~.
 \label{EQUATION::gausstheorem}
\end{equation}
%
%
	在方程(\ref{EQUATION::gausstheorem})中,$\textbf{n}$代表指向外部的表面法向量,d$S$是关于表面的积分,d$V$是关于体积的积分。


	\textbf{备注}: 小圆点 $\cdot$ 	表示两个向量的内积(\ref{EQUATION::innerProductVV})。在下面的书中,我们在所有的积分中,使用小圆点来表示我们计算的是一个向量\textbf{a}和 \textit{表面法向量}\textbf{n}的内积。请记住,小圆点表达的数学表达式与大圆点完全相同。
	


%==============================================================================
\end{document}
%==============================================================================
